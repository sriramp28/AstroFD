\documentclass[11pt]{article}
\usepackage{geometry}
\usepackage{amsmath}
\usepackage{amssymb}
\usepackage{graphicx}
\usepackage{hyperref}
\usepackage{longtable}
\usepackage{makecell}
\geometry{margin=1in}

\title{AstroFD: Relativistic Jet and SN-lite Solver}
\author{AstroFD Contributors}
\date{\today}

\begin{document}
\maketitle

\begin{abstract}
AstroFD is a finite-volume solver for relativistic hydrodynamics and
magnetohydrodynamics in flat spacetime and on fixed curved backgrounds.
The code targets relativistic jet simulations and SN-lite core-collapse
scenarios with configurable nozzle injection, GLM divergence cleaning,
causal dissipation, two-temperature relaxation, non-ideal MHD extensions,
H/He non-equilibrium chemistry, and parametric neutrino heating. This
report provides full governing equations, variable definitions, numerical
procedures, and validation workflows for reproducible research.
\end{abstract}

\section{Introduction}
Relativistic jets and core-collapse flows exhibit strong shocks, large
Lorentz factors, magnetic stresses, and stiff source terms. AstroFD is
built as a modular research code for controlled studies of these regimes.
It combines robust conservative discretizations with optional high-order
reconstruction and flexible physics modules.
Our design goal is to retain a clear separation between the conservative
hyperbolic update and optional source-term physics, so that each module
can be enabled, validated, and compared in isolation. This enables
systematic studies of jet stability, entrainment, and magnetization, and
controlled exploration of SN-lite heating and gravity prescriptions. The
code is not intended as a full radiation-hydrodynamic or general
relativistic spacetime solver, but instead as a flexible platform for
numerical experiments with well-defined approximations.

\section{Physical Background and Literature}
Relativistic jet modeling relies on SRHD and RMHD formulations for
ultrarelativistic flows, shocks, and magnetic stresses
\cite{marti1999, anile1989, komissarov1999}. GRHD/GRMHD is commonly
expressed using the Valencia 3+1 split \cite{banyuls1997, font2008}
with widely used astrophysical implementations such as HARM and ECHO
\cite{gammie2003, delzanna2007}. Divergence control in MHD is essential;
GLM cleaning provides a robust hyperbolic/parabolic mechanism
\cite{dedner2002, mignone2010}.

High-order reconstruction methods such as PPM and WENO improve accuracy in
smooth regions while maintaining stability across discontinuities
\cite{colella1984, jiang1996}. SSPRK time integration provides strong
stability-preserving updates for hyperbolic systems \cite{gottlieb2001}.

SN-lite physics uses parametric heating and cooling models motivated by
core-collapse supernova literature \cite{janka2001, janka2012}. Causal
relativistic dissipation follows Israel--Stewart-type formulations
\cite{israel1979, rezolla2013}. Resistive and non-ideal MHD extend ideal
models for controlled exploration of reconnection and drift physics
\cite{palenzuela2009}.
We emphasize that AstroFD is built to support reproducible comparisons
across numerical methods and physical closures. Each physics add-on is
implemented as a clearly isolated source term, enabling users to match
the mathematical form to published models and to quantify how the
numerical treatment impacts stability and accuracy. This framing is
particularly important for jet and SN-lite studies, where physical
uncertainties are large and sensitivity to numerical choices can be
substantial.

\subsection{Related Work and Positioning}
AstroFD is positioned as a modular research code emphasizing controlled
comparisons across physics closures and numerical options. In contrast to
production GRMHD codes such as HARM and ECHO \\cite{gammie2003, delzanna2007},
AstroFD prioritizes transparency and configurability over high-fidelity
transport or dynamic spacetime evolution. Its design follows established
finite-volume practice in SRHD/RMHD \\cite{marti1999, komissarov1999} while
providing explicit hooks for dissipation, chemistry, and SN-lite source
terms in a single framework.

\section{Notation and Variables}
We use units with $c = 1$ and metric signature $(-,+,+,+)$. Greek indices
represent spacetime components and Latin indices represent spatial
components. The 4-velocity is $u^\mu$, Lorentz factor is
$W = (1 - v^2)^{-1/2}$, and the specific enthalpy is
$h = 1 + e + p/\rho$. The ideal-gas EOS is $p = (\gamma - 1) \rho e$.

\begin{table}[ht]
\centering
\begin{tabular}{l l}
\hline
Symbol & Meaning \\
\hline
$\rho$ & rest-mass density \\
$e$ & specific internal energy \\
$p$ & thermal pressure \\
$u^\mu$ & 4-velocity \\
$v^i$ & 3-velocity \\
$W$ & Lorentz factor \\
$h$ & specific enthalpy \\
$T^{\mu\nu}$ & stress-energy tensor \\
$B^i$ & magnetic field (lab frame) \\
$b^\mu$ & magnetic 4-vector \\
$\alpha$ & lapse \\
$\beta^i$ & shift \\
$\gamma_{ij}$ & spatial metric \\
$\gamma$ & determinant of $\gamma_{ij}$ \\
$\psi$ & GLM cleaning scalar \\
$\Pi$ & bulk viscous pressure \\
$\pi^{\mu\nu}$ & shear stress \\
$q^\mu$ & heat flux \\
\hline
\end{tabular}
\caption{Notation used in the governing equations.}
\label{tab:notation}
\end{table}

\section{Model Assumptions and Scope}
Table \\ref{tab:model-assumptions} summarizes key assumptions across
physics modules to clarify applicability and limitations.

\begin{table}[ht]
\centering
\begin{tabular}{l p{0.7\linewidth}}
\hline
Module & Assumptions and Scope \\
\hline
SRHD/RMHD & Ideal-gas EOS; single-fluid; flat spacetime for SRHD/RMHD. \\
GRHD/GRMHD & Fixed analytic metric; no Einstein evolution; Valencia form. \\
GLM Cleaning & Hyperbolic/parabolic cleaning; no constrained transport. \\
Dissipation & Israel--Stewart relaxation with user-specified coefficients. \\
Two-Temperature & Single relaxation time; coarse-grained coupling. \\
Chemistry & Minimal H/He network; no metals or molecules. \\
Resistive/Non-Ideal & Explicit source corrections; may be stiff. \\
SN-lite Gravity & Newtonian point-mass or monopole potential. \\
SN-lite Heating & Parametric gain profiles; no transport solver. \\
Lightbulb Neutrinos & Simplified attenuation and cooling laws. \\
Radiation Coupling & Relaxation to imposed $T_{\rm rad}$. \\
\hline
\end{tabular}
\caption{Summary of assumptions and model scope.}
\label{tab:model-assumptions}
\end{table}

\section{Governing Equations}
\subsection{Overview}
We summarize the systems advanced by AstroFD in a unified conservation-law
form. Each system is expressed in covariant form, then cast into a 3+1
flux-conservative representation suitable for finite-volume updates.
Where applicable, we highlight how source terms are constructed for
gravity, dissipation, and heating/cooling.
\subsection{SRHD: Conservation Laws}
The covariant SRHD equations are
\begin{align}
\nabla_\mu (\rho u^\mu) &= 0, \\
\nabla_\mu T^{\mu\nu} &= 0,
\end{align}
with stress-energy tensor
\begin{equation}
T^{\mu\nu} = \rho h u^\mu u^\nu + p g^{\mu\nu}.
\end{equation}
In 3+1 form, the conserved variables are
\begin{align}
D &= \rho W, \\
S_i &= \rho h W^2 v_i, \\
\tau &= \rho h W^2 - p - D,
\end{align}
and the evolution equation is
\begin{equation}
\partial_t U + \partial_i F^i(U) = S(U),
\end{equation}
where $U = (D, S_i, \tau)$, $F^i$ are the conservative fluxes, and $S$
contains physical source terms (zero for flat SRHD).
In practice, we evolve $U$ on a Cartesian grid, reconstruct primitives
to faces, and compute fluxes using a chosen approximate Riemann solver.
The formulation is standard and widely used for relativistic jet
simulations, providing a consistent baseline for comparisons with MHD
and GR extensions.
\paragraph{Limitations and Assumptions.}
SRHD assumes a single-fluid ideal-gas EOS and neglects magnetic stresses.
All geometric source terms vanish in the flat-spacetime limit, so this
model is most appropriate for jet tests and baseline convergence studies.

\subsection{RMHD: Ideal MHD and GLM Cleaning}
Ideal RMHD satisfies
\begin{align}
\nabla_\mu (\rho u^\mu) &= 0, \\
\nabla_\mu T^{\mu\nu} &= 0, \\
\nabla_\mu {^*F}^{\mu\nu} &= 0,
\end{align}
with stress-energy tensor
\begin{equation}
T^{\mu\nu} = (\rho h + b^2) u^\mu u^\nu + \left(p + \frac{b^2}{2}\right) g^{\mu\nu} - b^\mu b^\nu,
\end{equation}
where $b^\mu$ is the magnetic 4-vector and $b^2 = b^\mu b_\mu$.
Divergence control is enforced by GLM cleaning
\cite{dedner2002, mignone2010}:
\begin{align}
\partial_t \mathbf{B} + \nabla \times \mathbf{E} + \nabla \psi &= 0, \\
\partial_t \psi + c_h^2 \nabla \cdot \mathbf{B} &= -c_p^2 \psi.
\end{align}
The GLM system is chosen for its ease of implementation and robustness
in finite-volume methods. It introduces a damped cleaning wave that
propagates divergence errors out of the domain without explicitly
constraining the magnetic field, and is compatible with the Riemann
solvers used here.
\paragraph{Limitations and Assumptions.}
AstroFD uses ideal RMHD with GLM cleaning and does not enforce a
constrained-transport update. Magnetic reconnection is therefore
numerically controlled unless resistive/non-ideal terms are enabled.

\subsection{GRHD/GRMHD: Valencia Formulation}
For fixed backgrounds, we adopt a 3+1 split with lapse $\alpha$, shift
$\beta^i$, and spatial metric $\gamma_{ij}$. The conservative update is
\begin{equation}
\partial_t (\sqrt{\gamma} U) + \partial_i (\sqrt{\gamma} F^i) = \sqrt{\gamma} S,
\end{equation}
with conserved variables
\begin{align}
D &= \rho W, \\
S_i &= (\rho h + b^2) W^2 v_i - \alpha b^0 b_i, \\
\tau &= (\rho h + b^2) W^2 - \left(p + \frac{b^2}{2}\right) - D - (\alpha b^0)^2.
\end{align}
Metric source terms are computed from the fixed background.
AstroFD assumes a fixed spacetime and does not evolve the Einstein
equations. This approximation is appropriate for jet propagation in
prescribed compact-object metrics and for test problems that emphasize
hydrodynamic and magnetic dynamics over spacetime evolution.
\paragraph{Limitations and Assumptions.}
The spacetime is static and analytic; dynamic spacetime evolution is out
of scope. Metric derivatives are computed analytically and inserted as
source terms, which is appropriate for fixed-background tests.

\subsubsection{Valencia Fluxes and Sources}
The Valencia formulation expresses the GRHD/GRMHD system as
\begin{equation}
\partial_t (\sqrt{\gamma} U) + \partial_i (\sqrt{\gamma} F^i) = \sqrt{\gamma} S,
\end{equation}
where $U = (D, S_j, \tau)$ and $F^i$ are
\begin{align}
F^i_D &= D (\alpha v^i - \beta^i), \\
F^i_{S_j} &= S_j (\alpha v^i - \beta^i) + \alpha \left(p + \frac{b^2}{2}\right)\delta^i{}_j - \alpha b^i b_j, \\
F^i_{\tau} &= \tau (\alpha v^i - \beta^i) + \alpha \left(p + \frac{b^2}{2}\right) v^i - \alpha b^0 b^i.
\end{align}
The source terms are
\begin{align}
S_D &= 0, \\
S_{S_j} &= \frac{\alpha}{2} T^{\mu\nu} \partial_j g_{\mu\nu}, \\
S_{\tau} &= \alpha \left(T^{\mu 0} \partial_\mu \ln \alpha - T^{\mu\nu} \Gamma^0_{\mu\nu}\right),
\end{align}
where $\Gamma^\alpha_{\mu\nu}$ are the Christoffel symbols.\\
AstroFD uses fixed analytic metrics (Minkowski, Schwarzschild, Kerr-Schild)
and computes these source terms from the prescribed background.

\paragraph{Magnetic 4-vector relations.}
In ideal RMHD, the lab-frame magnetic field $B^i$ and 4-vector $b^\mu$ are
related by
\begin{align}
b^0 &= W (B^i v_i), \\
b^i &= \frac{B^i}{W} + b^0 v^i,
\end{align}
and the ideal electric field satisfies $\mathbf{E} = -\mathbf{v} \times \mathbf{B}$.

\subsection{Causal Dissipation}
Israel--Stewart-type causal dissipation evolves bulk pressure $\Pi$,
shear stress $\pi^{\mu\nu}$, and heat flux $q^\mu$ with relaxation
\cite{israel1979, rezolla2013}:
\begin{align}
\tau_\Pi u^\alpha \nabla_\alpha \Pi + \Pi &= -\zeta \nabla_\alpha u^\alpha, \\
\tau_\pi u^\alpha \nabla_\alpha \pi^{\mu\nu} + \pi^{\mu\nu} &= 2 \eta \sigma^{\mu\nu}, \\
\tau_q u^\alpha \nabla_\alpha q^\mu + q^\mu &= -\kappa T \Delta^{\mu\nu} \nabla_\nu T.
\end{align}
These terms are applied as local source updates with configurable
relaxation times and coefficients. The intent is to provide a causal,
hyperbolic extension of Navier--Stokes suitable for relativistic flows,
while keeping the implementation modular for numerical experimentation.
\paragraph{Limitations and Assumptions.}
Transport coefficients are user-specified and spatially uniform unless
encoded through source terms. The model is intended for exploratory
studies rather than calibrated microphysical viscosity.

\subsection{Two-Temperature Relaxation}
Electron and ion internal energies evolve with relaxation time $\tau_{ei}$:
\begin{align}
\partial_t (\rho e_e) + \nabla \cdot (\rho e_e \mathbf{v}) &= \rho \frac{e_i - e_e}{\tau_{ei}}, \\
\partial_t (\rho e_i) + \nabla \cdot (\rho e_i \mathbf{v}) &= \rho \frac{e_e - e_i}{\tau_{ei}}.
\end{align}
This closure is useful for plasmas where electron and ion temperatures
remain out of equilibrium on dynamical timescales. AstroFD uses a
relaxation-time model that can be tuned to capture rapid coupling or
nearly decoupled limits.
\paragraph{Limitations and Assumptions.}
The two-temperature model uses a single relaxation timescale and does
not resolve kinetic-scale energy exchange. It is best interpreted as a
coarse-grained closure for parametric studies.

\subsection{H/He Non-Equilibrium Chemistry}
Species mass fractions $X_s$ satisfy
\begin{equation}
\partial_t (\rho X_s) + \nabla \cdot (\rho X_s \mathbf{v}) = \rho \dot{X}_s.
\end{equation}
The code uses collisional ionization and case-B recombination rates with
IMEX updates, following standard fits \cite{cen1992}.
We track a minimal H/He network to capture ionization energy exchange and
composition evolution without the cost of a full chemical solver. The
network is intended for qualitative studies and controlled comparisons
with analytic expectations rather than high-fidelity astrochemistry.
\paragraph{Limitations and Assumptions.}
The chemistry network is intentionally minimal and omits molecular
species and metal cooling. Rate coefficients are simplified fits and
should be interpreted as illustrative rather than definitive.

\subsection{Passive Tracers}
Passive tracers $C_k$ obey
\begin{equation}
\partial_t (\rho C_k) + \nabla \cdot (\rho C_k \mathbf{v}) = 0,
\end{equation}
with inlet and ambient values set by configuration.

\subsection{Resistive and Non-Ideal MHD}
Resistive MHD uses Ohm's law
\begin{equation}
\mathbf{E} + \mathbf{v} \times \mathbf{B} = \eta \mathbf{J},
\end{equation}
with $\mathbf{J} = \nabla \times \mathbf{B}$. Optional Hall and ambipolar
corrections are included as explicit terms for controlled experiments
\cite{palenzuela2009}.
These terms are implemented as additive source corrections, providing a
lightweight mechanism to probe reconnection and drift physics without
introducing full kinetic models. Because they can introduce stiffness,
their use is intended primarily for targeted experiments and validation.
\paragraph{Limitations and Assumptions.}
Non-ideal terms are applied explicitly and can be stiff; stability is
parameter-dependent and may require small timesteps or subcycling.

\subsection{SN-lite Gravity}
Newtonian gravity is applied as a source term with
\begin{equation}
\mathbf{g} = -G M(r) \frac{\mathbf{r}}{(r^2 + \epsilon^2)^{3/2}},
\end{equation}
using point-mass or monopole $M(r)$ options. Optionally, energy coupling
adds a pressure work term proportional to $\rho \mathbf{v} \cdot \mathbf{g}$.
The gravity model is Newtonian and used primarily for core-collapse-like
tests and parameter studies. For monopole gravity, the enclosed mass
profile can be computed from the density field and updated at a user-
specified cadence.
\paragraph{Limitations and Assumptions.}
The gravity model neglects relativistic corrections and assumes a static
potential. It is intended for SN-lite experiments and not precision
core-collapse modeling.

\subsection{SN-lite Gain Heating/Cooling}
The gain model applies pressure updates
\begin{equation}
\Delta p = (\gamma - 1) (Q^+ - Q^-) \, \mathcal{W}(r) \, \mathcal{S}(\rho, p, t) \, \Delta t,
\end{equation}
with a radial weight $\mathcal{W}(r)$ and scaling $\mathcal{S}$. Implemented
radial weights include:
\begin{align}
\mathcal{W}_{\mathrm{sph}} &= \left(1 + \xi^2\right)^{-1}, \quad \xi = \frac{r - r_0}{r_1}, \\
\mathcal{W}_{\mathrm{pow}} &= \left(\frac{r_0}{r}\right)^{p}, \\
\mathcal{W}_{\mathrm{exp}} &= \exp\left(-\frac{|r - r_0|}{r_1}\right), \\
\mathcal{W}_{\mathrm{gauss}} &= \exp\left(-\left(\frac{r - r_0}{r_1}\right)^2\right).
\end{align}
Density and pressure scalings use optional exponents, and a time decay
factor $\exp[-(t - t_0)/\tau]$ can be applied.
These gain-region prescriptions are intended to capture qualitative
heating behavior in the gain layer without detailed neutrino transport.
The model parameters are configurable and can be chosen to match
published lightbulb or gain-region setups.
\paragraph{Limitations and Assumptions.}
Heating and cooling are parameterized and spatially prescribed; no
neutrino transport is solved. Results should therefore be interpreted as
parametric sensitivity studies rather than predictive models.

\subsection{SN-lite Neutrino Lightbulb}
The lightbulb model applies a heating/cooling term using
\begin{align}
F_\nu &= \frac{L_\nu}{4 \pi r^2}, \\
\tau_{\mathrm{eff}} &= \max(\tau_0, \kappa \rho r), \\
Q^+ &= C_{\mathrm{abs}} F_\nu \epsilon_\nu e^{-\tau_{\mathrm{eff}}} \, \mathcal{W}(r), \\
Q^- &= C_{\mathrm{cool}} \rho T^{n},
\end{align}
with the same radial weight $\mathcal{W}(r)$ and optional time decay.
This lightbulb model provides a controllable energy deposition profile,
allowing systematic studies of shock revival thresholds and sensitivity
to heating parameters. It is not a substitute for full transport, but
is appropriate for controlled parametric exploration.
\paragraph{Limitations and Assumptions.}
The lightbulb model uses simplified attenuation and cooling laws and is
not a replacement for multi-group neutrino transport.

\subsection{Radiation--Plasma Coupling}
Optional radiation coupling relaxes the gas temperature toward $T_{\rm rad}$:
\begin{equation}
\frac{dT}{dt} = -\lambda (T - T_{\rm rad}),
\end{equation}
with $T \approx p/\rho$ in code units and a configurable coupling rate.
The coupling term is intentionally simple, intended to represent a
relaxation toward an imposed radiation temperature. It can be used to
study qualitative effects of radiative cooling/heating in jet or SN-lite
contexts without the overhead of radiative transfer.
\paragraph{Limitations and Assumptions.}
This relaxation does not solve radiation transport and assumes a single
imposed $T_{\\rm rad}$. It should be viewed as a qualitative coupling term.

\section{Numerical Methods}
\subsection{Overview}
AstroFD uses a standard finite-volume method with reconstruction,
Riemann solvers, and SSPRK time integration. The method is designed to be
robust for shocks and discontinuities while remaining extensible to
higher-order reconstructions.
\subsection{Finite-Volume Discretization}
We update cell-averaged conserved variables on a Cartesian mesh with
ghost zones. The semi-discrete update is
\begin{equation}
\frac{d U_{i,j,k}}{dt} = -\frac{F^x_{i+1/2,j,k} - F^x_{i-1/2,j,k}}{\Delta x}
-\frac{F^y_{i,j+1/2,k} - F^y_{i,j-1/2,k}}{\Delta y}
-\frac{F^z_{i,j,k+1/2} - F^z_{i,j,k-1/2}}{\Delta z} + S_{i,j,k}.
\end{equation}
The timestep is set by a CFL constraint
\begin{equation}
\Delta t = C_{\mathrm{CFL}} \min\left(\frac{\Delta x}{a_{\max}}, \frac{\Delta y}{a_{\max}}, \frac{\Delta z}{a_{\max}}\right),
\end{equation}
where $a_{\max}$ is the maximum signal speed.
Signal speeds are computed from the local state and the chosen physics
model. The CFL condition is enforced globally at each step to maintain
stability. For stiff source terms, additional subcycling is applied.

\subsection{Reconstruction}
Primitive variables are reconstructed to cell faces. MUSCL uses a limited
slope
\begin{equation}
\Delta q_i = \phi(\Delta q_{i-1/2}, \Delta q_{i+1/2}),
\end{equation}
with MC/minmod/van Leer limiters. PPM and WENO5 are available for higher
order accuracy \cite{colella1984, jiang1996}.
For MUSCL, slope limiting suppresses spurious oscillations near shocks.
PPM and WENO5 improve accuracy in smooth regions and are recommended for
studies requiring higher-order convergence.

\subsection{Riemann Solvers}
HLLE provides a robust baseline. HLLC is used for hydro, and HLLD/full
HLLD for RMHD \cite{toro1999, miyoshi2005}. The solver is selected at
runtime.
HLLE is the most robust choice for extreme conditions, while HLLC and
HLLD provide improved resolution of contact and Alfv\'en waves. The
selection can be tuned per experiment, and validation tests are provided
to assess solver behavior.

\subsection{Time Integration}
We support SSPRK2 and SSPRK3 integration \cite{gottlieb2001}.
SSPRK2 uses
\begin{align}
U^{(1)} &= U^n + \Delta t L(U^n), \\
U^{n+1} &= \frac{1}{2} U^n + \frac{1}{2} \left(U^{(1)} + \Delta t L(U^{(1)})\right).
\end{align}
SSPRK3 is available for improved stability at higher order. The choice
of RK order interacts with reconstruction order and should be validated
through convergence tests for high-accuracy studies.

\subsection{Source Term Coupling and IMEX}
Source terms are applied with operator splitting. For stiff terms, we
apply implicit relaxation with subcycled updates
\begin{align}
U^* &= U^n + \Delta t L(U^n), \\
U^{n+1} &= U^* + \Delta t S(U^{n+1}).
\end{align}
For stiff relaxation, subcycling ensures stable integration of source
terms without excessively limiting the global timestep. The number of
subcycles can be configured or computed adaptively.

\subsection{Primitive Recovery}
Primitive recovery solves for $(\rho, p, v_i)$ given the conservative
state. RMHD recovery uses iterative inversion with fallback strategies
\cite{noble2006}.
The recovery step is a common source of robustness issues in RMHD; the
code therefore applies multiple fallback paths and safety floors to
ensure physically admissible primitives.
\paragraph{Related Work.}
The numerical strategy follows common relativistic FV practice and is
compatible with methods used in PLUTO, HARM, and ECHO
\\cite{mignone2007, gammie2003, delzanna2007}. AstroFD emphasizes modular
comparisons between reconstruction and solver choices within a single
code base.

\subsection{Floors, Caps, and Stability Controls}
Density and pressure floors enforce positivity. Velocity caps limit $v$
close to $c$ to avoid numerical blow-up. These controls are configurable
and can be tightened during early testing.

\section{Algorithm Summary}
A single timestep proceeds as:
\begin{enumerate}
  \item Apply boundary conditions and nozzle inflow.
  \item Reconstruct primitives to faces (MUSCL/PPM/WENO).
  \item Solve Riemann problems at faces and compute fluxes.
  \item Update conserved variables with flux divergence.
  \item Apply source terms (gravity, GLM damping, dissipation, chemistry).
  \item Recover primitives and enforce floors/caps.
  \item Write diagnostics and outputs as configured.
\end{enumerate}
This algorithm is common to SRHD, RMHD, and GRHD/GRMHD modes, with
additional source terms for the relevant physics modules. The modular
structure allows direct comparisons between physics choices without
altering the core update loop.

\section{Boundary Conditions and Injection}
Jet inflow is imposed at the x-min boundary with configurable nozzle
profiles (top-hat, taper, parabolic) and optional perturbations to seed
instabilities. The nozzle includes a shear layer of configurable
thickness and optional magnetic fields (poloidal or toroidal) at the
inlet.

\section{Diagnostics}
Diagnostics include maximum Lorentz factor, inlet energy and momentum
fluxes, centerline profiles, divB statistics, SN shock radius and gain
mass, cocoon pressure, mixing layer thickness, and optional performance
counters. Diagnostics are written as CSV files in each run directory.
Diagnostics are designed to capture both numerical health (e.g., divB
levels, maximum Lorentz factor) and physically meaningful metrics (e.g.,
gain mass and heating efficiency). These outputs facilitate rapid
comparison across parameter studies and validation runs.
\paragraph{Related Work.}
Diagnostics are inspired by common jet and SN-lite metrics used in
parameter surveys, emphasizing reproducibility and rapid comparison of
model choices \\cite{janka2012}.

\section{Verification and Validation}
We track qualitative stability and quantitative thresholds in validation
scripts. Error norms are defined as
\begin{align}
L_1 &= \frac{1}{N} \sum_i |u_i - u_i^{\mathrm{ref}}|, \\
L_2 &= \left(\frac{1}{N} \sum_i (u_i - u_i^{\mathrm{ref}})^2\right)^{1/2}, \\
L_\infty &= \max_i |u_i - u_i^{\mathrm{ref}}|,
\end{align}
with an observed convergence rate
\begin{equation}
\mathcal{P} = \frac{\log(L(h_1)/L(h_2))}{\log(h_1/h_2)}.
\end{equation}
Validation suites cover reconstruction and limiter behavior, Riemann
solver checks, GR Kerr-Schild runs, orthonormal flux evaluation, restart
regression, and SN-lite tests (freefall, Sedov, stalled shock, and
lightbulb heating). These tests are automated in \texttt{tools/}.
Validation is ongoing and intended to support quantitative comparisons
across solvers and reconstructions. The current suite provides regression
coverage and sanity checks; future work will extend convergence studies
and standardized benchmark comparisons.
\paragraph{Related Work.}
Planned convergence studies and benchmark coverage align with standard
test suites used across relativistic hydro and MHD codes
\\cite{marti1999, miyoshi2005, font2008}.

\section{Configuration and Reproducibility}
All physics and numerics options are set through JSON configuration
files. Outputs are organized by timestamp under \texttt{results/}. The
code records diagnostics in CSV for post-processing and comparison.

\subsection{Recommended Configuration Table}
Table \ref{tab:config-summary} lists a compact reference of commonly
used configuration switches and suggested values for jets and SN-lite.

\begin{table}[ht]
\centering
\begin{tabular}{l l l}
\hline
Key & Jet default & SN-lite default \\
\hline
PHYSICS & rmhd & sn \\
RECON & ppm & muscl \\
RIEMANN & hlld\_full & hlle \\
RK\_ORDER & 3 & 2 \\
GLM\_CH, GLM\_CP & 1.0, 0.1 & n/a \\
TWO\_TEMPERATURE & true & true \\
CHEMISTRY\_ENABLED & true & true \\
RESISTIVE\_ENABLED & false & false \\
NEUTRINO\_ENABLED & false & true \\
\hline
\end{tabular}
\caption{Compact configuration reference for typical jet and SN-lite runs.}
\label{tab:config-summary}
\end{table}

\section{Example Workflows and Results}
This section includes example figures and a compact diagnostics summary
table generated from the latest validation run. The figures can be
regenerated with \texttt{tools/make\_doc\_figures.py}.

\subsection{Jet Propagation Snapshot}
\begin{figure}[ht]
\centering
\includegraphics[width=0.9\linewidth]{figures/jet_rho_midZ.png}
\caption{Jet density slice at mid-Z from the latest run.}
\end{figure}

\subsection{Centerline Lorentz Factor}
\begin{figure}[ht]
\centering
\includegraphics[width=0.9\linewidth]{figures/centerline_gamma.png}
\caption{Centerline Lorentz factor profile from the latest run.}
\end{figure}

\subsection{RMHD divB Control}
\begin{figure}[ht]
\centering
\includegraphics[width=0.9\linewidth]{figures/divb_max.png}
\caption{divB max over time from the latest RMHD run segment.}
\end{figure}

\subsection{SN-lite Shock Evolution}
\begin{figure}[ht]
\centering
\includegraphics[width=0.9\linewidth]{figures/sn_shock_radius.png}
\caption{SN-lite shock radius evolution from the latest run segment.}
\end{figure}

\subsection{Diagnostics Summary Table}
\begin{table}[h]
\centering
\begin{tabular}{l l}
\hline
Metric & Value \\ 
\hline
Max Lorentz factor & 1.048e+00 \\ 
Max signal speed & 9.934e-01 \\ 
Mean inlet flux (abs) & 1.727e-02 \\ 
divB max & N/A \\ 
divB rms & N/A \\ 
SN shock radius max & N/A \\ 
SN heating efficiency mean & N/A \\ 
\hline
\end{tabular}
\caption{Summary diagnostics from the latest run segment.}
\label{tab:validation-summary}
\end{table}


\appendix
\section{Valencia Derivation Details}
Let $n^\mu$ be the unit normal to constant-time slices and
$\gamma_{\mu\nu} = g_{\mu\nu} + n_\mu n_\nu$ be the spatial projector.
We decompose the 4-velocity as $u^\mu = W (n^\mu + v^\mu)$ with
$v^\mu n_\mu = 0$. The conserved energy density and momentum are
\begin{align}
E &= T^{\mu\nu} n_\mu n_\nu, \\
S_i &= -T^{\mu\nu} n_\mu \gamma_{\nu i}, \\
S_{ij} &= T^{\mu\nu} \gamma_{\mu i} \gamma_{\nu j}.
\end{align}
For ideal MHD, $b^\mu$ satisfies $b^\mu u_\mu = 0$, and the ideal electric
field is $\mathbf{E} = -\mathbf{v} \times \mathbf{B}$. The Valencia system
then follows from projecting $\nabla_\mu T^{\mu\nu} = 0$ along $n^\nu$ and
within the spatial hypersurface. The code uses analytic fixed metrics,
so $\alpha$, $\beta^i$, $\gamma_{ij}$, and Christoffel symbols are
computed analytically for each supported background.

\section{Configuration Parameters}
Tables in this section list the default configuration keys from
\texttt{utils/settings.py}. Units are code units unless stated otherwise.
\small
\setlength{\tabcolsep}{4pt}
\begin{longtable}{p{0.32\linewidth} p{0.15\linewidth} p{0.45\linewidth}}
\caption{Configuration parameter defaults (utils/settings.py).}\\label{tab:config-full}\\
\hline
Key & Default & Notes \\
\hline
\endfirsthead
\hline
Key & Default & Notes \\
\hline
\endhead
ADAPTIVITY\\ENABLED & false & \\
ADAPTIVITY\\MODE & nested\_static & \\
ADAPTIVITY\\REFINEMENT & 2 & \\
ADAPTIVITY\\REGION & null & \\
ADAPTIVITY\\SUBCYCLES & null & \\
AMBIPOLAR\\COEFF & 0 & \\
AMBIPOLAR\\ENABLED & false & \\
ASSERTS & false & \\
B0 & 0 & \\
BULK\_ZETA & 0 & \\
B\_INIT & none & \\
CFL & 0.35 & \\
CHECKPOINT\\EVERY & 0 & \\
CHECK\_NAN\\EVERY & 0 & \\
CHEMISTRY\\ENABLED & false & \\
CHEM\_ENERGY\\SCALE & 1 & \\
CHEM\_RATE\\SCALE & 1 & \\
CHEM\_TMIN\_K & 1 & \\
CHEM\_TUNIT\_K & 10000 & \\
CHEM\_X & 0.76 & \\
CHEM\_X\_HEIII\\AMB & 0 & \\
CHEM\_X\_HEIII\\NOZZLE & null & \\
CHEM\_X\_HEII\\AMB & 0 & \\
CHEM\_X\_HEII\\NOZZLE & null & \\
CHEM\_X\_HII\\AMB & 0 & \\
CHEM\_X\_HII\\NOZZLE & null & \\
CHEM\_Y & 0.24 & \\
COOLING\\ENABLED & false & \\
COOLING\\LAMBDA & 0 & \\
DEBUG & false & \\
DIAG\_COCOON\\ENABLED & false & \\
DIAG\_COCOON\\TRACER\_IDX & 0 & \\
DIAG\_COCOON\\TRACER\_MAX & 1 & \\
DIAG\_COCOON\\TRACER\_MIN & 0.05 & \\
DIAG\_MIXING\\ENABLED & false & \\
DIAG\_MIXING\\MAX & 0.95 & \\
DIAG\_MIXING\\MIN & 0.05 & \\
DIAG\_MIXING\\TRACER\_IDX & 0 & \\
DIAG\_MIXING\\X\_FRAC & 0.5 & \\
DIAG\_PLANE\\ENABLED & false & \\
DIAG\_PLANE\_X & null & \\
DIAG\_SPECTRA\\ENABLED & false & \\
DIAG\\STRUCTURE\\ENABLED & false & \\
DIAG\\STRUCTURE\\MAX\_LAG & 16 & \\
DISSIPATION\\ADVECT & true & \\
DISSIPATION\\CAP\_FRAC & 0.5 & \\
DISSIPATION\\ENABLED & false & \\
DISSIPATION\\MODEL & israel\_stewart & \\
DISSIPATION\\SUBCYCLES & 1 & \\
DISSIPATION\\SUBCYCLES\\AUTO & true & \\
DISSIPATION\\SUBCYCLES\\MAX & 32 & \\
DISSIPATION\\SUBCYCLE\\FRACTION & 0.5 & \\
EOS\_GAMMA\\TABLE\_RHO & null & \\
EOS\_GAMMA\\TABLE\_VAL & null & \\
EOS\_MODE & gamma & \\
EOS\\PIECEWISE\\GAMMA & null & \\
EOS\\PIECEWISE\\RHO & null & \\
EOS\_TABLE\\LOG & true & \\
ETA\_RHO & 0.05 & \\
GAMMA & 1.66667 & \\
GAMMA\_JET & 6 & \\
GLM\_CH & 1 & \\
GLM\_CP & 0.1 & \\
GR\_MASS & 1 & \\
GR\_METRIC & minkowski & \\
GR\_SPIN & 0 & \\
HALL\_COEFF & 0 & \\
HALL\_ENABLED & false & \\
HALO\\EXCHANGE & blocking & \\
HEATING\_RATE & 0 & \\
HEAT\_KAPPA & 0 & \\
HYPERRESIST\\COEFF & 0 & \\
HYPERRESIST\\ENABLED & false & \\
JET\_CENTER & null & \\
JET\_RADIUS & 0.12 & \\
JOULE\_HEAT\\EFF & 0 & \\
JOULE\_HEAT\\ENABLED & false & \\
KINETIC\_CAP & 0 & \\
KINETIC\\COEFF & 0 & \\
KINETIC\\EFFECTS\\ENABLED & false & \\
KINETIC\\MODEL & shear & \\
LIMITER & mc & \\
Lx & 1 & \\
Ly & 1 & \\
Lz & 1 & \\
NEUTRINO\_ABS\\COEFF & 0 & \\
NEUTRINO\\COOL\_COEFF & 0 & \\
NEUTRINO\\COOL\_TEXP & 6 & \\
NEUTRINO\\ENABLED & false & \\
NEUTRINO\_EPS & 1 & \\
NEUTRINO\\KAPPA & 0 & \\
NEUTRINO\_LUM & 0 & \\
NEUTRINO\\MODEL & lightbulb & \\
NEUTRINO\_R0 & 0 & \\
NEUTRINO\_R1 & 0 & \\
NEUTRINO\_TAU\\FLOOR & 0 & \\
NEUTRINO\\TIME\_T0 & 0 & \\
NEUTRINO\\TIME\_TAU & 0 & \\
NG & 2 & \\
NONIDEAL\_MHD\\ENABLED & false & \\
NOZZLE\\PERTURB & null & \\
NOZZLE\\PROFILE & top\_hat & \\
NOZZLE\_TURB & true & \\
NUMBA\_NUM\\THREADS & null & \\
NX & 128 & \\
NY & 96 & \\
NZ & 96 & \\
N\_TRACERS & 0 & \\
OMP\_NUM\\THREADS & null & \\
ORTHONORMAL\\FLUX & true & \\
OUT\_EVERY & 50 & \\
PERF\_ENABLED & false & \\
PERF\_EVERY & 10 & \\
PERF\_RESET\\EVERY & 0 & \\
PHYSICS & hydro & \\
PRINT\_EVERY & 10 & \\
P\_AMB & null & \\
P\_EQ & 0.01 & \\
P\_MAX & 1 & \\
RADIATION\\COEFF & 0 & \\
RADIATION\\COUPLING\\ENABLED & false & \\
RADIATION\\COUPLING\\TARGET & gas & \\
RADIATION\_T\\RAD & 0 & \\
RECON & muscl & \\
RELAX\_TAU & 0.1 & \\
RELAX\_TAU\\BULK & null & \\
RELAX\_TAU\\HEAT & null & \\
RELAX\_TAU\\SHEAR & null & \\
RESISTIVE\\ENABLED & false & \\
RESISTIVITY & 0 & \\
RESTART\_PATH & null & \\
RESTART\\STRICT & false & \\
RESULTS\\UNIQUE & false & \\
RHO\_AMB & 1 & \\
RIEMANN & hlle & \\
RK\_ORDER & 2 & \\
RMHD\_INIT & uniform & \\
RMHD\_RIEMANN\\LEFT & null & \\
RMHD\_RIEMANN\\RIGHT & null & \\
RMHD\_RIEMANN\\X0 & 0.5 & \\
SAVE\_RUN\\CONFIG & true & \\
SHEAR\_ETA & 0 & \\
SHEAR\_THICK & 0.02 & \\
SN\_COMP\_AMB\\VALUES & null & \\
SN\_COMP\\NAMES & null & \\
SN\_COMP\\NOZZLE\\VALUES & null & \\
SN\_COOLING\\RATE & 0 & \\
SN\_COOLING\\TEXP & 0 & \\
SN\_COOLING\\TREF & 1 & \\
SN\_EOS\_GAMMA & 1.33333 & \\
SN\_GAIN\_POW & 2 & \\
SN\_GAIN\\RADIUS & 0.2 & \\
SN\_GAIN\\WIDTH & 0.1 & \\
SN\_GRAVITY\\BINS & 64 & \\
SN\_GRAVITY\\CENTER & null & \\
SN\_GRAVITY\\ENABLED & false & \\
SN\_GRAVITY\\ENERGY & false & \\
SN\_GRAVITY\_G & 1 & \\
SN\_GRAVITY\\MASS & 1 & \\
SN\_GRAVITY\\MODEL & point\_mass & \\
SN\_GRAVITY\\SOFTEN & 0.01 & \\
SN\_GRAVITY\\UPDATE\_EVERY & 1 & \\
SN\_HEATING\\ENABLED & false & \\
SN\_HEATING\\MODEL & gain\_spherical & \\
SN\_HEATING\_P\\EXP & 0 & \\
SN\_HEATING\\RATE & 0 & \\
SN\_HEATING\\RHO\_EXP & 0 & \\
SN\_HEATING\\TIME\_T0 & 0 & \\
SN\_HEATING\\TIME\_TAU & 0 & \\
SN\_INIT & uniform & \\
SN\_P\_IN & 1 & \\
SN\_P\_OUT & 0.01 & \\
SN\_RHO\_IN & 1 & \\
SN\_RHO\_OUT & 0.1 & \\
SN\_SEDOV\_DP & 1 & \\
SN\_SEDOV\\RADIUS & 0.05 & \\
SN\_SHOCK\_P\\RATIO & 1.2 & \\
SN\_SHOCK\_P\\REF & null & \\
SN\_SPHERE\\RADIUS & 0.2 & \\
TEI\_TAU & 0.5 & \\
TE\_AMB & null & \\
TE\_NOZZLE & null & \\
TI\_AMB & null & \\
TI\_NOZZLE & null & \\
TRACER\_AMB\\VALUES & null & \\
TRACER\_NAMES & null & \\
TRACER\\NOZZLE\\VALUES & null & \\
TURB\_PAMP & 0 & \\
TURB\_VAMP & 0.02 & \\
TWO\\TEMPERATURE & false & \\
T\_END & 0.25 & \\
VX\_AMB & 0 & \\
VY\_AMB & 0 & \\
VZ\_AMB & 0 & \\
V\_MAX & 0.999 & \\
\hline
\end{longtable}
\normalsize
\setlength{\tabcolsep}{6pt}


\section{Paper Template Checklist}
\subsection{Recommended Figures}
\begin{itemize}
  \item Jet density slice and Lorentz factor centerline profile.
  \item RMHD divB control over time for representative runs.
  \item Shock radius evolution for SN-lite scenarios.
  \item Resolution study comparing MUSCL/PPM/WENO convergence.
  \item GR metric validation plots (e.g., Kerr-Schild test case).
\end{itemize}

\subsection{Recommended Tables}
\begin{itemize}
  \item Configuration summary for each experiment class.
  \item Diagnostics summary (max Gamma, divB, shock radius).
  \item Convergence errors for scheme comparisons.
\end{itemize}

\subsection{Results Checklist}
\begin{itemize}
  \item State the EOS, reconstruction, Riemann solver, and time integrator.
  \item Report CFL, grid resolution, and runtime stability settings.
  \item Provide divB statistics for RMHD/GRMHD runs.
  \item Report shock radius and heating efficiency for SN-lite runs.
  \item Include convergence or resolution-sensitivity tests.
  \item Archive configs and diagnostics alongside plots.
\end{itemize}

\section{Limitations and Future Work}
Current limitations include static nested refinement (single rank),
limited GRMHD benchmark coverage, and incomplete GPU acceleration for
solver kernels. Planned improvements include expanded EOS options,
quantitative convergence tests, and additional validation problems.

\bibliographystyle{unsrt}
\bibliography{astrofd_refs}

\end{document}
