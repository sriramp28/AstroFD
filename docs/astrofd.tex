\documentclass[11pt]{article}
\usepackage{geometry}
\usepackage{amsmath}
\usepackage{amssymb}
\usepackage{hyperref}
\geometry{margin=1in}

\title{AstroFD: Relativistic Jet and SN-lite Solver}
\author{AstroFD Contributors}
\date{\today}

\begin{document}
\maketitle

\begin{abstract}
AstroFD is a finite-volume solver for relativistic hydrodynamics and
magnetohydrodynamics in flat spacetime and on fixed curved backgrounds.
The code targets relativistic jet simulations and SN-lite core-collapse
scenarios with configurable nozzle injection, GLM divergence cleaning,
causal dissipation, two-temperature relaxation, non-ideal MHD extensions,
H/He non-equilibrium chemistry, and parametric neutrino heating. This
report documents the governing equations, numerical methods, physical
modules, diagnostics, and validation tests.
\end{abstract}

\section{Introduction}
Relativistic jets and core-collapse flows exhibit strong shocks, large
Lorentz factors, magnetic stresses, and stiff source terms. AstroFD is
built as a modular research code for controlled studies of these regimes.
It combines robust conservative discretizations with optional high-order
reconstruction and flexible physics modules.

\section{Code Architecture}
The solver is organized around a conservative update in Cartesian
coordinates. Physics options are selected by configuration and include
SRHD/RMHD in flat spacetime and GRHD/GRMHD in fixed metrics. Source terms
for dissipation, chemistry, and SN-lite physics are applied in operator
split steps with optional IMEX subcycling for stiff terms. Outputs are
organized by timestamp and include diagnostics and snapshot fields.

\section{Governing Equations}
\subsection{Special Relativistic Hydrodynamics}
We evolve conservative variables $U = (D, S_i, \tau)$ derived from
primitive variables $(\rho, v_i, p)$ with an ideal-gas EOS
$ p = (\gamma - 1) \rho e$. The conservative variables are
\begin{align}
D &= \rho W, \\
S_i &= \rho h W^2 v_i, \\
\tau &= \rho h W^2 - p - D,
\end{align}
with specific enthalpy $h = 1 + e + p/\rho$ and Lorentz factor
$W = (1 - v^2)^{-1/2}$. Fluxes follow standard SRHD expressions
\cite{marti1999}.

\subsection{Special Relativistic MHD}
RMHD extends SRHD with magnetic fields $\mathbf{B}$ and a GLM scalar
$\psi$ to control $\nabla \cdot \mathbf{B}$ \cite{dedner2002}. We adopt
an ideal RMHD formulation with magnetic pressure and tension, and a GLM
system of the form
\begin{align}
\partial_t \mathbf{B} + \nabla \times \mathbf{E} + \nabla \psi &= 0, \\
\partial_t \psi + c_h^2 \nabla \cdot \mathbf{B} &= -c_p^2 \psi,
\end{align}
with hyperbolic cleaning speed $c_h$ and damping $c_p$. The GLM terms are
added to the conservative update and damped through source terms.

\subsection{GRHD/GRMHD on Fixed Backgrounds}
We implement the Valencia formulation of GRHD/GRMHD on analytic metrics
(Minkowski, Schwarzschild, Kerr-Schild) with optional orthonormal-frame
flux evaluation for improved stability near strong curvature
\cite{banyuls1997, font2008, gammie2003}. Metric source terms are computed
from the chosen background and included in the conservative update.

\subsection{Causal Dissipation}
Israel--Stewart-type causal dissipation evolves bulk, shear, and heat
flux variables with relaxation times and source terms
\cite{israel1979, rezolla2013}. Stiff relaxation is integrated with
subcycled IMEX steps to maintain stability.

\subsection{Two-Temperature Relaxation}
A two-temperature model evolves electron and ion internal energies with
relaxation time $\tau_{ei}$. The coupling is parametrized by the user and
can be enabled or disabled at runtime.

\subsection{H/He Non-Equilibrium Chemistry}
A simplified H/He network tracks ion fractions with source terms for
ionization and recombination. Cooling and heating are applied
consistently with the ion state. Species are stored as passive scalars
advected with the flow.

\subsection{Passive Tracers}
Passive tracers are advected as additional conserved scalars, initialized
with configurable nozzle and ambient values. These tracers enable mixing
and entrainment diagnostics without affecting the dynamics.

\subsection{Resistive and Non-Ideal MHD}
Resistive RMHD adds Ohmic diffusion with resistivity $\eta$. Optional
Hall, ambipolar, and hyper-resistive terms are included as explicit
corrections for controlled experiments.

\subsection{SN-lite Source Terms}
SN-lite physics includes Newtonian gravity (point-mass or monopole),
parametric gain-region heating/cooling, and a lightbulb neutrino source
term for approximate energy deposition \cite{janka2001}. These are
intended for controlled tests rather than full radiation transport.

\section{Numerical Methods}
\subsection{Finite-Volume Discretization}
We update cell-averaged conserved variables on a Cartesian mesh with
ghost zones. Fluxes are computed at faces from reconstructed states and
a Riemann solver. Source terms are applied in split fashion.

\subsection{Reconstruction}
Available reconstructions include MUSCL (with MC, minmod, or van Leer
limiters), PPM \cite{colella1984}, and WENO5 \cite{jiang1996}. The choice
is set at runtime via configuration.

\subsection{Riemann Solvers}
HLLE provides a robust baseline. HLLC is available for hydro, and HLLD
and full HLLD for RMHD \cite{toro1999, miyoshi2005}. The full HLLD solver
captures additional wave structure compared to reduced variants.

\subsection{Time Integration}
SSPRK2 and SSPRK3 are supported \cite{gottlieb2001}. Causal dissipation
and stiff source terms use subcycled IMEX updates for stability.

\subsection{Floors, Caps, and Recovery}
Density and pressure floors and velocity caps protect against unphysical
states. RMHD primitive recovery includes fallback paths for robustness.

\section{Boundary Conditions and Injection}
Jet inflow is imposed at the x-min boundary with configurable nozzle
profiles (top-hat, taper, parabolic) and optional perturbations to seed
instabilities. The nozzle includes a shear layer of configurable
thickness and optional magnetic fields (poloidal or toroidal) at the
inlet.

\section{Diagnostics}
Diagnostics include maximum Lorentz factor, inlet energy and momentum
fluxes, centerline profiles, divB statistics, SN shock radius and gain
mass, cocoon pressure, mixing layer thickness, and optional performance
counters. Diagnostics are written as CSV files in each run directory.

\section{Validation}
Validation suites cover reconstruction and limiter behavior, Riemann
solver checks, GR Kerr-Schild runs, orthonormal flux evaluation, restart
regression, and SN-lite tests (freefall, Sedov, stalled shock, and
lightbulb heating). These tests are automated in \texttt{tools/}.

\section{Configuration and Reproducibility}
All physics and numerics options are set through JSON configuration
files. Outputs are organized by timestamp under \texttt{results/}. The
code records diagnostics in CSV for post-processing and comparison.

\subsection{Recommended Configuration Table}
Table \ref{tab:config-summary} lists a compact reference of commonly
used configuration switches and suggested values for jets and SN-lite.

\begin{table}[h]
\centering
\begin{tabular}{l l l}
\hline
Key & Jet default & SN-lite default \\\\
\hline
PHYSICS & rmhd & sn \\\\
RECON & ppm & muscl \\\\
RIEMANN & hlld\_full & hlle \\\\
RK\_ORDER & 3 & 2 \\\\
GLM\_CH, GLM\_CP & 1.0, 0.1 & n/a \\\\
TWO\_TEMPERATURE & true & true \\\\
CHEMISTRY\_ENABLED & true & true \\\\
RESISTIVE\_ENABLED & false & false \\\\
NEUTRINO\_ENABLED & false & true \\\\
\hline
\end{tabular}
\caption{Compact configuration reference for typical jet and SN-lite runs.}
\label{tab:config-summary}
\end{table}

\section{Example Workflows and Results}
This section provides placeholders for figures and tables that can be
filled with results from the validation suite or production runs. The
placeholders compile without external files and can be replaced with
published plots once available.

\subsection{Jet Propagation Snapshot}
\begin{figure}[h]
\centering
\fbox{\rule{0pt}{2.2in}\rule{0.9\linewidth}{0pt}}
\caption{Example jet density slice at a representative time. Replace this
placeholder with a plot generated by \texttt{tools/quickview.py}.}
\end{figure}

\subsection{RMHD divB Control}
\begin{figure}[h]
\centering
\fbox{\rule{0pt}{2.2in}\rule{0.9\linewidth}{0pt}}
\caption{Example divB statistics over time for an RMHD jet. Replace this
placeholder with diagnostics from \texttt{divb.csv}.}
\end{figure}

\subsection{SN-lite Shock Evolution}
\begin{figure}[h]
\centering
\fbox{\rule{0pt}{2.2in}\rule{0.9\linewidth}{0pt}}
\caption{Example SN-lite shock radius evolution from
\texttt{sn\_diagnostics.csv}.}
\end{figure}

\section{Limitations and Future Work}
Current limitations include static nested refinement (single rank),
limited GRMHD benchmark coverage, and incomplete GPU acceleration for
solver kernels. Planned improvements include expanded EOS options,
quantitative convergence tests, and additional validation problems.

\bibliographystyle{unsrt}
\bibliography{astrofd_refs}

\end{document}
